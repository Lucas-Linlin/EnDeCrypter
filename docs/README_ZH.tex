% !TEX program = xelatex
\documentclass{article}
\usepackage[UTF8]{ctex}
\usepackage{listings}
\lstset{
    basicstyle=\ttfamily\small,
    backgroundcolor=\color{gray!10},
    frame=single,
    numbers=left,
    numberstyle=\tiny\color{gray},
    breaklines=true,
    tabsize=4
}
\usepackage{amsmath}
\usepackage{tikz}
\usepackage{mathptmx}
\usetikzlibrary{calc, shapes, arrows, angles}
\usepackage[colorlinks=true, linkcolor=black, urlcolor=blue, citecolor=green]{hyperref}

\title{加解密器 V0.3.2 用户手册}
\author{LinLUCAS}
\date{\today}
\begin{document}
\maketitle
\newpage
\tableofcontents
\newpage
\section{引言}
感谢您安装加解密器。在使用之前,请务必阅读此用户手册。

加解密器是一个成立于2025年的项目,旨在帮助人们传递秘密消息和记录(不)重要的机密。
\section{用户协议与隐私条款}

\subsection{用户协议}

加解密器(以下简称“本软件”)是一个开源、免费的软件。

本软件是且仅是一个娱乐用软件,不可用于商业、政治等正式用途。

本软件开源,但您不能将本软件以任何形式进行免费或收费的二次分发。

您可以参考本软件的源代码进行自己的软件创作。但引用超过10行代码需联系我们获得许可。

\subsection{隐私条款}

我们不会窃取您的任何个人信息,包括但不限于姓名、性别、银行账号等。

我们会获取您输入的文本。请确保其中不含任何敏感信息。如发生信息泄露,我们不能保证您的信息的绝对安全。

\section{使用指南}

\subsection{下载与安装}

访问\href{https://github.com/Lucas-Linlin/EnDeCrypter/releases/latest}{这个链接}以获取最新编译的程序,下载 \lstinline|EnDeCrypter.exe|。下载完成后,请直接双击运行即可。

如被Windows报为病毒,请参考\href{https://github.com/Lucas-Linlin/EnDeCrypter/}{这里的README.md}解决。

\subsection{加密与解密}

\subsubsection{加密}

要加密一条信息,您首先需要确定一个方法和一个密码。

方法,即加密的方法,由一个或几个不区分大小写的字母‘a’、‘b’或‘c’构成。方法的长度必须大于等于1,但每个字母不必都出现。它决定了您的信息将会经过怎样的(物理性)处理。其中,‘a’代表转换字符(ABCDEFGH $\rightarrow$ BADCFEHG),‘c’代表倒置字符串,‘b’代表随机插入字符。

密码,就是一串纯数字。它必须为一个正整数。

当您(随便)想好了一个方法和一个密码以后,您就可以把信息、方法和密码输入进对应的输入框中。点击“加密”按钮,您就可以在下面的输出框中看见输出了!

\subsubsection{解密}

解密与加密很相似,把对应的信息、方法、密码输入进输入框中,再点击“解密”按钮即可解密。

**注意,错误的方法或密码一定会导致解密失败!所以不要妄图破译密码或加密后的机密消息!**

\subsection{信息安全性}

由于采用了先进的算法,使用本软件进行加密的信息很难被破译。但要注意的是,消息被破译的可能性并非为零(事实上没有几个加密算法能真正做到这点)。

在此,我们还是要提示您:保存好您的方法和密码。加密有风险,使用需谨慎!

\section{结语}

再次感谢您安装加解密器。祝您使用愉快!
\end{document}